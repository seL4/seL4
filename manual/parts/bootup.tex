%
% Copyright 2014, General Dynamics C4 Systems
%
% SPDX-License-Identifier: GPL-2.0-only
%

\chapter{\label{ch:bootup}System Bootstrapping}

\section{Initial Thread's Environment}

The seL4 kernel creates a minimal boot environment for the initial thread, which
 is started at priority \texttt{seL4\_MaxPrio} and maximum control priority \texttt{seL4\_MaxPrio}.
This environment consists of the initial thread's TCB, CSpace and VSpace,
consisting of frames that contain the userland image (code/data of the initial
thread) and the IPC buffer.

On the MCS kernel, the initial thread is configured with a round-robin scheduling context
with \texttt{CONFIG\_BOOT\_THREAD\_TIME\_SLICE} milliseconds timeslice.
Without MCS, all threads including the initial thread are scheduled round-robin with
\texttt{CONFIG\_TIMER\_TICK\_MS} $*$ \texttt{CONFIG\_TIME\_SLICE} timeslices.

The initial thread's CSpace consists of exactly one CNode
which contains capabilities to the initial
thread's own resources as well as to all available global resources.
The CNode size can be configured at compile time (default is $2^{12}$
slots), but the guard is always chosen so that the CNode resolves exactly
the number of bits in the architecture (32 bits or 64 bits). This means, the
first slot of the CNode has CPTR 0x0, the second slot has CPTR 0x1 etc.

The first 15 slots (or 14 slots if not MCS) contain specific capabilities as listed in
\autoref{tab:cnode_content}.

\begin{table}[htb]
  \begin{center}
    \caption{Initial thread's CNode content.}
    \label{tab:cnode_content}
    \begin{tabularx}{\textwidth}{lX}
      \toprule
      Enum Constant & Capability \\
      \midrule
      \texttt{seL4\_CapNull}                & null \\
      \texttt{seL4\_CapInitThreadTCB}       & initial thread's TCB \\
      \texttt{seL4\_CapInitThreadCNode}     & initial thread's CNode \\
      \texttt{seL4\_CapInitThreadVSpace}    & initial thread's VSpace \\
      \texttt{seL4\_CapIRQControl}          & global IRQ controller (see \autoref{sec:interrupts}) \\
      \texttt{seL4\_CapASIDControl}         & global ASID controller (see \autoref{ch:vspace}) \\
      \texttt{seL4\_CapInitThreadASIDPool}  & initial thread's ASID pool (see \autoref{ch:vspace}) \\
      \texttt{seL4\_CapIOPort}              & global I/O port cap, null cap if unsupported
\ifxeightsix
(see \autoref{sec:ioports})
\fi
\\
      \texttt{seL4\_CapIOSpace}             & global I/O space cap, null cap if unsupported
\ifxeightsix
(see \autoref{sec:iospace})
\fi
\\
      \texttt{seL4\_CapBootInfoFrame}       & BootInfo frame (see \autoref{ch:bootup:bootinfo}) \\
      \texttt{seL4\_CapInitThreadIPCBuffer} & initial thread's IPC buffer (see \autoref{sec:messageinfo}) \\
      \texttt{seL4\_CapDomain}              & domain cap (see \autoref{sec:domains}) \\
      \texttt{seL4\_CapSMMUSIDControl}      & global Arm SMMU SID controller, null cap if unsupported (see \autoref{sec:smmuv2}) \\
      \texttt{seL4\_CapSMMUCBControl}       & global Arm SMMU CB controller, null cap if unsupported (see \autoref{sec:smmuv2}) \\
      \texttt{seL4\_CapInitThreadSC}        & initial thread's scheduling context (MCS only) \\
      \bottomrule
    \end{tabularx}
  \end{center}
\end{table}

\section{\label{ch:bootup:bootinfo}BootInfo Frame}

CNode slots with CPTR \texttt{seL4\_NumInitialCaps} (defined in the seL4
userland library) and above are filled dynamically during
bootstrapping. Their exact contents depend on the userland image size,
platform configuration (devices) etc. In order to tell the initial thread
which capabilities are stored where in its CNode, the kernel provides
a \emph{BootInfo Frame} which is mapped into the initial thread's address
space. The mapped address is chosen by the kernel and given to the initial
thread via a CPU register.

The BootInfo Frame contains the C struct described in
\autoref{tab:bootinfo_struct}.
It is defined in the seL4 userland library. Besides talking about
capabilities, it also informs the initial thread about
the current platform's configuration.

The type \texttt{seL4\_SlotRegion} is a C struct
which contains \texttt{start} and \texttt{end} slot CPTRs. It denotes a region
of slots in the initial thread's CNode, starting with CPTR \texttt{start} and with
\texttt{end} being the CPTR of the first slot after the region ends, i.e.\
\texttt{end - 1} points to the last slot of the region.

\begin{table}[htb]
  \begin{center}
    \caption{BootInfo struct.}
    \label{tab:bootinfo_struct}
    \begin{tabularx}{\textwidth}{llX}
      \toprule
      Field Type & Field Name & Description \\
      \midrule
      \texttt{seL4\_Word}           & \texttt{extraLen}                & length of additional bootinfo information in bytes \\
      \texttt{seL4\_NodeId}         & \texttt{nodeID}                  & node ID \\
      \texttt{seL4\_Word}           & \texttt{numNodes}                & number of nodes \\
      \texttt{seL4\_Word}           & \texttt{numIOPTLevels}           & number of I/O page-table levels (-1 if CONFIG\_IOMMU unset) \\
      \texttt{seL4\_IPCBuffer*}     & \texttt{ipcBuffer}               & pointer to the initial thread's IPC buffer \\
      \texttt{seL4\_SlotRegion}     & \texttt{empty}                   & empty slots (null caps) \\
      \texttt{seL4\_SlotRegion}     & \texttt{sharedFrames}            & reserved \\
      \texttt{seL4\_SlotRegion}     & \texttt{userImageFrames}         & frames containing the userland image \\
      \texttt{seL4\_SlotRegion}     & \texttt{userImagePaging}         & userland-image paging structure caps \\
      \texttt{seL4\_SlotRegion}     & \texttt{ioSpaceCaps}             & I/O space capabilities for Arm SMMU \\
      \texttt{seL4\_SlotRegion}     & \texttt{extraBIPages}            & frames backing additional bootinfo information \\
      \texttt{seL4\_Uint8}          & \texttt{initThreadCNodeSizeBits} & CNode size ($2^n$ slots) \\
      \texttt{seL4\_Word}           & \texttt{initThreadDomain}        & domain of the initial thread (see \autoref{sec:domains}) \\
      \texttt{seL4\_SlotRegion}     & \texttt{schedcontrol}            & seL4\_SchedControl capabilities, one for each node (MCS only). \\
      \texttt{seL4\_SlotRegion}     & \texttt{untyped}                 & untyped-memory capabilities \\
      \texttt{seL4\_UntypedDesc[]}  & \texttt{untypedList}             & array of information about each untyped \\
      \bottomrule
    \end{tabularx}
  \end{center}
\end{table}

Depending on the architecture and platform there might be additional pieces of boot
information. If \texttt{extraLen} is greater than zero, then 4K after the start of bootinfo
is a region of extraLen bytes containing additional bootinfo structures. Each chunk starts
with a \texttt{seL4\_BootInfoHeader}, described in \autoref{tab:bi_header_struct}, that
describes what the chunk is and how long it is, where the length includes the header. The
length can be used to skip over chunks that you do not understand. The only generally
defined chunk type is \texttt{SEL4\_BOOTINFO\_HEADER\_PADDING} and describes an empty
chunk that has no data, any other types are platform or architecture specific. The
\texttt{extraBIPages} slot region gives the frames capabilities for the pages that make up
the additional boot info region.

\begin{table}[htb]
  \begin{center}
    \caption{BootInfoHeader struct.}
    \label{tab:bi_header_struct}
    \begin{tabular}{lll}
      \toprule
      Field Type & Field Name & Description \\
      \midrule
      \texttt{seL4\_Word}  & \texttt{id}  & Identifier indicating the contents of the chunk \\
      \texttt{seL4\_Uint8} & \texttt{len} & Length in bytes of the chunk \\
      \bottomrule
    \end{tabular}
  \end{center}
\end{table}

The capabilities in \texttt{userImageFrames} are
ordered such that the first capability references the first frame of the
userland image and so on.
The capabilities in \texttt{userImagePaging} are ordered in descending order
of paging structure size. Within a given paging structure size, capabilities are
ordered by the virtual address at which the corresponding objects are mapped
into the initial thread's address space.

It is up to userland to infer the virtual address of frames referenced by
the capabilities in \texttt{userImageFrames} and the virtual address and
types of paging structures
referenced by the capabilities in \texttt{userImagePaging}.
Userland typically has a way of finding out to which virtual addresses its
code and data is mapped (e.g.\ in GCC, with the standard linker script, the
symbols \texttt{\_\_executable\_start} and \texttt{\_end} are available).
Additionally, the initial thread can assume that its address space is virtually
contiguous, and is made up of the smallest frames available on the architecture.
It's also assumed that the initial thread knows which paging structures are
available on the architecture it's running on.
This, along with knowledge of how capabilities in \texttt{userImageFrames} and
\texttt{userImagePaging} are ordered, is sufficient information for userland to infer
the virtual address of each
frame capability, and the virtual address and type of each paging structure capability.

Untyped memory is given in no particular order. The array entry
\texttt{untypedList[i]} stores the untyped-memory information of
the i-th untyped cap of the slot region \texttt{untyped}. Therefore, the array
length is at least \texttt{untyped.end - untyped.start}. The actual length is
hardcoded in the kernel and irrelevant to the reader of the array. The untyped
memory information is stored in a \texttt{seL4\_UntypedDesc} struct, described
in \autoref{tab:untyped_desc_struct}, and details the address, size and kind of
the memory backing the untyped. This allows userland to infer physical memory
addresses of retyped frames and use them to initiate DMA transfers when no
IOMMU is available. The kernel makes no guarantees about certain sizes of untyped
memory being available.

\begin{table}[htb]
  \begin{center}
    \caption{seL4\_UntypedDesc struct}
    \label{tab:untyped_desc_struct}
    \begin{tabular}{lll}
      \toprule
      Field Type & Field Name & Description \\
      \midrule
      \texttt{seL4\_Word}  & \texttt{paddr}    & physical base address of the untyped object \\
      \texttt{seL4\_Uint8} & \texttt{sizeBits} & size ($2^n$ bytes) of the untyped object \\
      \texttt{seL4\_Uint8} & \texttt{isDevice} & is this untyped a device or not (see \autoref{sec:kernmemalloc}) \\
      \texttt{seL4\_Uint8[]} & \texttt{padding} & manual padding so final struct is a multiple of the word size \\
      \bottomrule
    \end{tabular}
  \end{center}
\end{table}

If the platform has an seL4-supported IOMMU, \texttt{numIOPTLevels} contains
the number of IOMMU-page-table levels. This information is needed by userland
when constructing an IOMMU address space (IOSpace). If there is no IOMMU
support, \texttt{numIOPTLevels} is \texttt{0}.

On Arm if the platform has any available SMMU units the capabilities for them
will be described by the \texttt{ioSpaceCaps} slot region. The mapping of a
capability from this region to a specific SMMU is platform specific.

\ifxeightsix
\section{Boot Command-line Arguments}

On IA-32, seL4 accepts boot command-line arguments which are passed to the
kernel via a multiboot-compliant bootloader (e.g.\ GRUB, syslinux). Multiple
arguments are separated from each other by whitespace. Two forms of arguments
are accepted:
(1) key-value arguments of the form ``key=value'' and (2) single keys of the
form ``key''. The value field of the key-value form may be a string, a decimal
integer, a hexadecimal integer beginning with ``0x'', or an integer list where
list elements are separated by commas.
Keys and values can't have any whitespace in them and there can be no
whitespace before or after an ``='' or a comma either.
Arguments are listed in \autoref{tab:bootargs} along with their default values (if left unspecified).


\begin{table}[htb]
    \caption{IA-32 boot command-line arguments.}
        \begin{tabularx}{\textwidth}{lXX}
            \toprule
              Key & Value & Default \\
            \midrule
            \texttt{console\_port} &
            I/O-port base of the serial port that the kernel prints to
            (if compiled in debug mode) &
            0x3f8 \\
            \texttt{debug\_port} &
            I/O-port base of the serial port that is used for kernel debugging
            (if compiled in debug mode) &
            0x3f8 \\
            \texttt{disable\_iommu} &
            none &
            The IOMMU is enabled by default on VT-d-capable platforms \\
            \bottomrule
        \end{tabularx}
    \label{tab:bootargs}
\end{table}
\fi
